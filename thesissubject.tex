\documentclass[10pt,a4paper]{article}
\usepackage[utf8]{inputenc}
\usepackage[francais]{babel}
\usepackage[T1]{fontenc}
\usepackage{amsmath}
\usepackage{amsfonts}
\usepackage{amssymb}
\usepackage{graphicx}
\usepackage{lmodern}
\author{Thomas Dupriez}
\begin{document}

% PRESENTATION DETAILLEE DU SUJET DE THESE

% (DE 2 A 5 PAGES)

% Ce sujet détaillé peut etre le meme que le résumé du sujet de thèse qui est construit en utilisant l'application web de candidature en ligne, mais ce n'est pas obligatoire.

% Vous etes libre, dans cette partie, de rédiger le sujet de la thèse comme vous l'entendez, d'ajouter une ou deux illustrations et d'ajouter quelques références bibliographiques.

% Vous enregistrerez ce sujet détaillé au format PDF et le déposerez sur l'interface de canidature ADUM pour le 11/03/2018. Le document original, daté et signé par le directeur de thèse et vous-meme doit etre déposé au secrétariat de votre département pour le 18/03/2018.

% Les deux experts de la discipline consultés pour évaluer votre dossier de candidature auront l'ensemble du dossier (sujet résumé, sujet détaillé et cursus scolaire).
% Les membres de la commission de sélection s'appuieront sur le sujet résumé et les avis des experts et des directeurs de département.

% Le sujet complet doit etre présenté au directeur du laboratoire, au directeur de l'Ecole doctorale et le cas échéant, au directeur du département d'enseignement.
% Leurs avis datés et signés sur votre candidature sont requis pour le 18/03/2018.


\end{document}
